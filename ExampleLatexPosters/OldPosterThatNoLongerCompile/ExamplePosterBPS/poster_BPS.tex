% Template file for an a0 landscape poster.
% Written by Graeme, 2001-03 based on Norman's original microlensing
% poster.
%
% See discussion and documentation at
% <http://www.astro.gla.ac.uk/users/norman/docs/posters/> 
%
% $Id: poster-template-landscape.tex,v 1.2 2002/12/03 11:25:46 norman Exp $


% Default mode is landscape, which is what we want, however dvips and
% a0poster do not quite do the right thing, so we end up with text in
% landscape style (wide and short) down a portrait page (narrow and
% long). Printing this onto the a0 printer chops the right hand edge.
% However, 'psnup' can save the day, reorienting the text so that the
% poster prints lengthways down an a0 portrait bounding box.
%
% 'psnup -w85cm -h119cm -f poster_from_dvips.ps poster_in_landscape.ps'

\documentclass[a0]{a0poster}
% You might find the 'draft' option to a0 poster useful if you have
% lots of graphics, because they can take some time to process and
% display. (\documentclass[a0,draft]{a0poster})

\pagestyle{empty}
\setcounter{secnumdepth}{0}

% The textpos package is necessary to position textblocks at arbitary 
% places on the page.
\usepackage[absolute]{textpos}

% Graphics to include graphics. Times is nice on posters, but you
% might want to switch it off and go for CMR fonts.
% Greg commented this out 
% \usepackage{graphics,wrapfig,times}

% Greg's additions 
\input{math_def}
\input{ca_def}

\def\Bold#1{{\bf #1}}
\def\Block#1#2{{\bf #1:} {#2}\\ }

\def\bfcite#1{{\bf\cite{#1}}}

\usepackage{epsfig}
\usepackage{amsmath}
\usepackage{amssymb}

% These colours are tried and tested for titles and headers. Don't
% over use color!
\usepackage{color}
\definecolor{DarkBlue}{rgb}{0.1,0.1,0.5}
\definecolor{Red}{rgb}{0.9,0.0,0.1}

% see documentation for a0poster class for the size options here
\let\Textsize\normalsize
\def\Head#1{\noindent\hbox to \hsize{\hfil{\LARGE\color{DarkBlue} #1}}\bigskip}
\def\LHead#1{\noindent{\LARGE\color{DarkBlue} #1}\bigskip}
\def\CHead#1{\begin{center} {\LARGE\color{DarkBlue} #1} \end{center}}
\def\Subhead#1{\noindent{\large\color{DarkBlue} #1}\bigskip}

%\def\Title#1{\noindent{\VeryHuge\color{Red} #1}}
%\def\Title#1{\begin{center} {\sc \VeryHuge\color{Red} #1} \end{center} }
%\def\Title#1{\begin{center} {\sc \Huge\color{Red} #1} \end{center} }
\def\Title#1{\begin{center} {\sc \Huge\color{DarkBlue} #1} \end{center} }

% Set up the grid
%
% Note that [40mm,40mm] is the margin round the edge of the page --
% it is _not_ the grid size. That is always defined as 
% PAGE_WIDTH/HGRID and PAGE_HEIGHT/VGRID. In this case we use
% 23 x 12. This gives us three columns of width 7 boxes, with a gap of
% width 1 in between them. 12 vertical boxes is a good number to work
% with.
%
% Note however that texblocks can be positioned fractionally as well,
% so really any convenient grid size can be used.
%
% GREG: 5.3 0.6 5.3 0.6 5.3 0.6 5.3 
% GREG: 0 5.9 11.8 17.7 
%

\TPGrid[40mm,40mm]{23}{12}      % 3 cols of width 7, plus 2 gaps width 1

\parindent=0pt
\parskip=0.5\baselineskip

\begin{document}

% Understanding textblocks is the key to being able to do a poster in
% LaTeX. In
%
%    \begin{textblock}{wid}(x,y)
%    ...
%    \end{textblock}
%
% the first argument gives the block width in units of the grid
% cells specified above in \TPGrid; the second gives the (x,y)
% position on the grid, with the y axis pointing down.

% You will have to do a lot of previewing to get everything in the 
% right place.

% This gives good title positioning for a portrait poster.
% Watch out for hyphenation in titles - LaTeX will do it
% but it looks awful.
\begin{textblock}{23}(0,0)
\Title{markov chain models of instantaneously-coupled calcium channels}
\end{textblock}

\begin{textblock}{23}(0,0.4)
\CHead{Gregory D.~Smith, Vien D.~Nguyen, Jie (Vivian) Zhang\\[0.2in]
Department of Applied Science, College of William and Mary, Williamsburg, VA 23187} 
%({\tt greg@as.wm.edu, marco@as.wm.edu, jxgrof@wm.edu})
\end{textblock}

% Uni logo in the top right corner. A&A in the bottom left. Gives a
% good visual balance, but you may want to change this depending upon
% the graphics that are in your poster.
%\begin{textblock}{2}(0,11.5)
%Your logo here
%\includegraphics{/usr/local/share/images/AandA.epsf}
%\end{textblock}
%\begin{textblock}{2}(21.2,0)
%Another logo here
%\resizebox{2\TPHorizModule}{!}{\includegraphics{/usr/local/share/images/GUVIu/GUVIu.eps}}
%\end{textblock}

\begin{textblock}{5.3}(0,0.7)
\CHead{--- Introduction ---}
The effect of feedback inhibition on retinogeniculate transmission is studied
using a minimal network model that includes $\sim$10--100 representative
thalamocortical (TC) neurons of the dorsal lateral geniculate nucleus (dLGN)
reciprocally connected with thalamic reticular (RE) neurons of the
perigeniculate nucleus (PGN).  \end{textblock}

\begin{textblock}{5.3}(0,2.2)
\CHead{--- TC- \& RE-like IFB Models --- }
Network simulations are based on 
the {\it integrate-and-fire-or-burst} 
(IFB) model (Smith et al.~2000, Rinzel 1980)  
constructed by adding a slow variable to a classical
integrate-and-fire neuron model.  The slow variable, $h$, represents the
inactivation of the low-threshold \Ca\ conductance, $I_T$.  
\bnea
C \ddt{V} &=& - I_L - I_T \nonumber \\
  \ddt{h} &=& \left\{ \begin{array}{cc}
                      -h/\tau_h^-  & \( V > V_h \) \\
                      (1-h)/\tau_h^+ & \( V < V_h \)
                      \end{array}
              \right. 
\nonumber 
\enea
A spike occurs whenever the membrane potential reaches a threshold
($V_\theta$) for firing an action potential, \[ V(t) = V_\theta
\Longrightarrow V(t^+) = V_{reset} \] and the current balance equation
includes two constant conductance leakage currents $I_L = g_{KL} \( V -
V_{KL} \) + g_{NL} \( V - V_{NL} \)$ and the low-threshold \Ca\ current, $I_T
= g_T \mu \{ V - V_h \} h \( V - V_T \)$, where $\mu$ and $h$ characterize
the activation and inactivation of $I_T$, respectively. 

A subtle change in parameters converts an IFB model designed to
reproduce TC neuron responses into an IFB model that responds like the
inhibitory RE neurons of the PGN.  
\begin{center} 
\includegraphics[width=20cm,height=10cm,clip=]{FigMarco01.eps} \end{center}
The distinction is the relationship between $V_h$ ({\it vertical red line}),
the threshold for $I_T$, and $V_L$ ({\it filled circle}), the resting
membrane potential. In the TC-like IFB model, $I_{T}$ is inactivated at rest
($V_L > V_h$), while the RE-like IFB model is primed to burst in response to
depolarizing input ($V_{L} < V_h$).
\end{textblock}

%=========== SECOND COLUMN ==========

\begin{textblock}{5.3}(5.9,1.2)
\CHead{--- Network Equations --- }
In the dLGN/PGN network, each TC and RE neuron is described
by an IFB model.  The TC cells receive spontaneous or visually-driven
excitatory synaptic currents ($I_{RET}$) and feedback inhibition from
associated RE cells ($I_{GABA}$).  The RE cells receive excitatory input from
the TC cells ($I_{AMPA}$) and are inhibited by neighboring RE cells
($I_{GABA_{RE}}$).  
\bnea 
C \ddt {V_{TC}^i} &=& - I_{MEM_{TC}}^i - I_{RET}^i - I_{GABA}^i \nonumber \\
C \ddt {V_{RE}^i} &=& - I_{MEM_{RE}}^i - I_{AMPA}^i-I_{GABA_{RE}}^i  \nonumber 
\enea 
\[ I_{syn}^i = g_{syn} \sum_{j} w_{ji} s_{j} \( V^{i} - V_{syn} \) 
\quad \mbox{where} \quad w_{ij} = w_{ij} \( | x_i - x_j | \) . \] 
where the synaptic footprint ($w_{ij}$) 
is either exponentially decaying or all-to-all 
and alpha-function postsynaptic conductances ($s_j$) are triggered by spiking
of the presynaptic neuron (Golomb et al. 1996). 

A gamma process modulated at rate $\rho(x,t)$ gives the event times for
excitatory postsynaptic conductances representing spontaneous or
visually-stimulated retinal input calculated via convolution of a
luminance profile with the spatio-temporal receptive field of one class
(e.g., Xon) of retinal ganglion cell.  Drifting grating-like retinal drive
takes the form \[ \rho \( t\) = \rho_{DC} + \rho_{AC} \cos \( 2 \pi k x - 2
\pi f t \) \] with spatial and temporal frequency $k$ and $f$, respectively.
\end{textblock}

\begin{textblock}{5.3}(5.9,6.0)
\begin{center}
%\includegraphics[width=18cm,height=17cm]{FigThal13.eps}
\end{center}
\end{textblock}

\begin{textblock}{7.3}(0,9.0)
\CHead{--- Neuromodulation \& Tuning of Synaptic Strengths --- }
\end{textblock}

\begin{textblock}{3.33}(0,9.7)
Subcortical neuromodulation of TC and RE neuron state is
simulated by changing the potassium leakage conductances, $g_{KL}^{TC}$ and
$g_{KL}^{RE}$.  This conductance affects $V_L$, the
de-inactivation level of $I_T$, and burst vs.\ tonic response mode of
both cell types.  Synaptic strengths in the\\ dLGN/PGN model were chosen so
in the {\it Sleep state} rhythmic bursting occurs in the presence of the
30 Hz spontaneous drive of the dark-adapted retina.  
An increase of $g_{KL}^{RE}$ and decrease of
$g_{KL}^{TC}$ puts the network into the {\it Awake State} 
and rhythmic bursting does not persist.  \end{textblock}

\begin{textblock}{3.33}(3.93,9.4)
\begin{center}
%\includegraphics[width=15cm,height=15cm,clip=]{RhythBurstSleepAwakeState.eps}
\end{center}
\end{textblock}

%==============  THIRD COLUMN ================
\begin{textblock}{5.3}(11.9,1.2)
\CHead{--- Full-Field Stimulation --- } 
\end{textblock}

\begin{textblock}{2.6}(11.9,1.5)
\CHead{\large High Contrast \& Low Freq} 
\begin{center}
%\includegraphics[width=11cm,height=13cm]{InhibitionOnOffModulatedInput.eps}

{\small $\rho_{DC}$ = 30 Hz, $\rho_{AC}$ = 60 Hz, $f$ = 1.53 Hz }
\end{center}
\end{textblock}

\begin{textblock}{2.6}(14.5,1.5)
\CHead{\large Low Contrast \& High Freq} 
\begin{center}
%\includegraphics[width=11cm,height=13cm]{LowContrastInhibitionOnOffModulatedInput.eps}

{\small $\rho_{DC}$ = 70 Hz, $\rho_{AC}$ = 20 Hz, $f$ = 4 Hz }
\end{center}
\end{textblock}

\begin{textblock}{5.3}(11.9,4.4)
Feedback-inhibition from RE cells leads to
post-inhibitory rebound bursting ({\it red}) in TC cells that can dominate
network throughput. Conversely, TC cells faithfully transmit retinal input in
the absence of feedback inhibition. 
Although the network is ``awake'' and does not
support rhythmic bursting in the 
presence of spontaneous retinal drive (see {\bf
Neuromodulation}), episodes of irregular
bursting can be triggered by visually-driven retinal input. 
\end{textblock}

\begin{textblock}{5.3}(11.8,5.6)
\CHead{--- Temporal Frequency Analysis --- } 
\end{textblock}

\begin{textblock}{2.65}(11.8,6.0)
\begin{center}
%\includegraphics[width=9cm,height=9cm]{ResponseMeasuresHighContrast.eps}
\end{center}
\end{textblock}

\begin{textblock}{2.65}(14.45,6.0)
\begin{center}
%\includegraphics[width=9cm,height=9cm,clip=]{ResponseMeasuresLowContrast.eps}
\end{center}
\end{textblock}

\begin{textblock}{5.3}(11.8,7.8)
Fourier fundamental response ($F_1$) plotted as a function of stimulation
frequency ($f$) during {\bf Full-Field Stimulation} with 
maximum retinal input rate of $\rho_{DC}+\rho_{AC}$ = 90 Hz.  {\it
High Contrast}: At high temporal frequencies, TC cell fundamental responses
is proportional to retinal input. At low frequencies, TC cell fundamental
response follows RE cell (as opposed to retinal) activity. 
{\it Low Contrast}: 
The higher value of $\rho_{DC}$ leads to RE activity  
that persists at high stimulation frequencies. 

\end{textblock}

\begin{textblock}{9.1}(8,9.0)
\CHead{--- Firing Rate Model, RE State \& Transmission --- } 
RE cell neuromodulation ($g_{KL}^{RE}$) 
influences the input/output properties of the
dLGN/PGN model by changing RE response 
({\it silent}, {\it burst}, 
{\it gap}, {\it tonic}) 
to TC cell excitation. Network throughput also
depends on TC neuron state ($g_{KL}^{TC}$, not shown). 
\end{textblock}

\begin{textblock}{0.2}(7.7,10.2)
RE\\[1in] TC\\[1in] RET
\end{textblock}

\begin{textblock}{2.6}(7.8,9.8)
\begin{center}
RE burst 

%\includegraphics[width=10cm,height=10cm]{DriftingGratingGkl0.008.eps}

$g_{KL}^{RE}$ large (hyperpolarized)
\end{center}
\end{textblock}
\begin{textblock}{2.6}(10.2,9.8)
\begin{center}
RE gap 

%\includegraphics[width=10cm,height=10cm]{DriftingGratingGkl-0.018.eps}
\end{center}
\end{textblock}
\begin{textblock}{2.6}(12.6,9.8)
\begin{center}
RE tonic 

%%%%%%%%%\includegraphics[width=10cm,height=10cm]{DriftingGratingGkl-0.022.eps}
\end{center}
\end{textblock}
\begin{textblock}{2.6}(15.0,9.8)
\begin{center}
RE tonic 

%\includegraphics[width=10cm,height=10cm]{DriftingGratingGkl-0.026.eps}

$g_{KL}^{RE}$ small (depolarized) 
\end{center}
\end{textblock}

% ================== FOURTH COLUMN ============

\begin{textblock}{5.3}(17.7,0.7)
\CHead{--- Drifting Grating Stimulation --- } 
\end{textblock}

\begin{textblock}{2.6}(17.7,1.1)
\CHead{\Large IFB Network} 
\end{textblock}

\begin{textblock}{2.6}(20.4,1.1)
\CHead{\Large FR Network} 
\end{textblock}

\begin{textblock}{2.6}(17.7,1.4)
\begin{center}
$k$ = 1 Hz

%\includegraphics[width=10cm,height=10cm,clip=]{DriftingGratingWavelength1.0Footprint0.1.eps}
\end{center}
\end{textblock}

\begin{textblock}{2.6}(17.7,3.3)
\begin{center}
$k$ = 2 Hz

%\includegraphics[width=10cm,height=10cm,clip=]{DriftingGratingWavelength0.5Footprint0.1.eps}
\end{center}
\end{textblock}

\begin{textblock}{2.6}(17.7,5.2)
\begin{center}
$k$ = 10 Hz

%\includegraphics[width=10cm,height=10cm,clip=]{DriftingGratingWavelength0.1Footprint0.1.eps}
\end{center}
\end{textblock}

\begin{textblock}{5.3}(17.7,7.2)
\CHead{--- Spatial Frequency \& Transmission --- } 
In the {\it IFB Network} model and a corresponding 
firing-rate reduction ({\it FR Network}, see Smith et al.~2001),
the spatial frequency of drifting grating-like retinal drive influences the 
activity of RE cells, feedback inhibition, and throughput.
Here the temporal frequency of 4 Hz is fixed, while the spatial
frequency is varied (1--3 Hz). 
The exponential synaptic footprint has length constant 
of 1/10 the spatial domain.
At low spatial frequency the RE cells respond to periodic excitation 
from TC cells with bursts, periodically inhibit the TC cells, and 
influence network throughput.  
At high spatial frequency the RE cells fail to respond 
to periodic excitation from TC cells and the
network throughput follows the retinal drive. 
Note that the drift rate of the grating is slower
at high spatial frequency, so this phenomenon is to be distinguished
from the high temporal frequency cut-off 
in RE cell activity shown in {\bf Temporal Frequency Analysis}. 
\end{textblock}

\begin{textblock}{2.6}(20.5,1.4)
\begin{center} $k$ = 1 Hz 
%\includegraphics[width=10cm,height=10cm]{DriftingGratingSF1.eps}
\end{center}
\end{textblock}

\begin{textblock}{2.6}(20.5,3.3)
\begin{center} $k$ = 2 Hz 
%\includegraphics[width=10cm,height=10cm]{DriftingGratingSF2.eps}
\end{center}
\end{textblock}

\begin{textblock}{2.6}(20.5,5.2)
\begin{center} $k$ = 3 Hz

%\includegraphics[width=10cm,height=10cm]{DriftingGratingSF3.eps}
\end{center}
\end{textblock}


\begin{textblock}{5.3}(17.7,9.9)
\CHead{--- References --- } 
{\small 
J. Rinzel.  Models in neurobiology.  In: R.H. Enns et al., {\it Nonlinear
phenomena in physics and biology,} 1980, pp.~347--367. 

D. Golomb, X.J. Wang, and J. Rinzel. 
Propagation of spindle waves in a thalamic Slice model. 
{\it J.~Neurophysiol.} 72(2):750--769, 1996. 

G.D. Smith and C.L. Cox and S.M. Sherman and J. Rinzel.  Fourier analysis of
sinusoidally-driven thalamocortical relay neurons and a minimal
integrate-and-fire-or-burst model.  J. Neurophys., 83(1):588-610, 2000.

G.D. Smith and C.L. Cox and S.M. Sherman and J. Rinzel.  Spike-frequency
adaptation in sinusoidally-driven thalamocortical relay neurons.  Thalamus and
Related Systems. 11:1-22, 2001.
}
\end{textblock}


%================== BOTTOM ===================
\begin{textblock}{23}(0,12.1)
\begin{center}
Supported by NSF grants IBN 0228273 and MCB 0133132 to GDS, the Thomas F. and
Kate Miller Jeffress Memorial Trust, and computational facilities at W\&M
enabled by NSF and Sun Microsystems.  GDS acknowledges conversations with
J Rinzel, S Coombes, K Srinivasan, and P Brewer.  \end{center} \end{textblock}

\end{document}
