
\exercise[(heat equation with discontinuous data)]{9.3}

\begin{enumerate}

\item Modify \verb+heat_CN.m+ to solve the heat equation for
$-1\leq x \leq 1$ with step function  initial data
\begin{equation} \label{9.3a}
u(x,0) = \begin{choices}  1 \when x<0\\  0 \when x\geq 0. \end{choices}
\end{equation} 
With appropriate Dirichlet boundary conditions, the exact solution is
\begin{equation} \label{9.3b}
u(x,t) = \half \, \text{erfc} \left(x / \sqrt{4 \kappa t}\right),
\end{equation} 
where erfc is the complementary error function
\[
\text{erfc}(x) = \frac{2}{\sqrt{\pi}} \int_x^\infty e^{-z^2}\,dz.
\]

\begin{enumerate}
\item
Test this routine $m=39$ and $k = 4h$.  Note that there is an initial rapid
transient decay of the high wave numbers that is not captured well with this
size time step.

\item
How small do you need to take the time step to get reasonable results?
For a suitably small time step, explain why you get much better results by
using $m=38$ than $m=39$.  What is the observed order of accuracy as $k\goto
0$ when $k = \alpha h$ with $\alpha$ suitably small and $m$ even?

\end{enumerate}

\item Modify \verb+heat_trbdf2.m+ (see Exercise~9.2)
to solve the heat equation for
$-1\leq x \leq 1$ with step function  initial data as above.
Test this routine using $k=4h$ and estimate the order of accuracy as $k\goto
0$ with $m$ even.  Why does the TR-BDF2 method work better than
Crank-Nicolson?

\end{enumerate} 
