

\exercise[(Lax-Richtmyer stability of leapfrog as a one-step method)]{10.10}
Consider the leapfrog method for the advection equation $u_t + au_x=0$ on
$0\leq x\leq 1$ with periodic boundary conditions.  From the von Neumann
analysis of Example 10.4 we expect this method to be stable for $|ak/h|< 1$.  
However, the Lax Equivalence theorem as stated in Section 9.5 only
applies to 1-step (2-level) methods.  
The point of this exercise is to show that the 3-level leapfrog method can
be interpreted as a 1-step method to which the Lax Equivalence theorem
applies.

The leapfrog method $U^{n+1} = U^{n-1} + 2kAU^n$ can be rewritten as
\eqlex{a}
\bcm U^{n+1}\\ U^n\ecm = \bcm 2kA&I \\ I&0\ecm \bcm U^n\\U^{n-1}\ecm,
\end{equation}
which has the form $V^{n+1} = BV^n$.

\begin{enumerate}
\item
Show that the matrix $B$ defined by \eqnex{a} has $2(m+1)$ eigenvectors of
the form
\eqlex{b}
\bcm g_p^- u^p\\ u^p\ecm, \quad \bcm g_p^+ u^p\\ u^p\ecm, \quad\text{for}~
p=1,~2,~\ldots,~m+1,  
\end{equation}
where $u^p\in\reals^{m+1}$ are the eigenvectors of $A$ given by (10.12) 
and $g_p^{\pm}$ are the two roots of a quadratic equation.  Explain how this
quadratic equation relates to (10.34) (what values of $\xi$ are relevant for
this grid?)

What are the eigenvalues of $B$?

\item
Show that if
\eqlex{c}
|ak/h|< 1
\end{equation} 
then the eigenvalues of $B$ are distinct with magnitude equal to 1.

\item
The result of part (b) is not sufficient to prove that leapfrog is
Lax-Ricthmyer stable.  The matrix $B$ is not normal and the matrix of right
eigenvectors $R$ with columns given by \eqnex{b} is not unitary.  By (D.8)
in Appendix D we have
\eqlex{d}
\|B^n\|_2 \leq \|R\|_2 \|R^{-1}\|_2 = \kappa_2(R).
\end{equation}
To prove uniform power boundedness and stability we must show that the
condition number of $R$ is uniformly bounded as $k\goto 0$ provided
\eqnex{c} is satisfied.

Prove this by the following steps:
\begin{enumerate}
\item Let 
\eqlex{e}
U = \frac 1 {\sqrt{m+1}} \left[u^1~~u^2~\cdots~ u^p\right]
\in\reals^{(m+1)\times (m+1)}
\end{equation}
be an appropriately scaled right eigenvector matrix of $A$.  Show that with
this scaling, $U$ is a unitary matrix:  $U^HU=I$.

\item Show that the right eigenvector matrix of $B$ can be written as
\eqlex{f}
R = \bcm UG^- & UG^+\\ U&U\ecm
\end{equation}
where $G^{\pm} = \text{diag}(g_1^\pm,~\ldots,~g_{m+1}^\pm)$.  

\item Show that if $x = \bcm x\\y\ecm \in\reals^{2(m+1)}$ has $\|z\|_2=1$
then $\|Rz\|_2 \leq C$ for some constant independent of $m$, and hence
$\|R\|_2 \leq C$ for all $k$.  (It is fairly easy to show this with
$C=2\sqrt{2}$ and with a bit more work that in fact $\|R\|_2=2$ for all
$k$.)

\item Let
\eqlex{g}
L = \bcm G^-U^H & U^H\\ G^+ U^H & U^H\ecm.
\end{equation}
Show that $LR$ is a diagonal matrix and hence $R^{-1}$ is a diagonal scaling
of the matrix $L$.  Determine $R^{-1}$.

\item Use the previous result to show that
\eqlex{h}
\|R^{-1}\|_2 \leq \frac{C}{1-\nu^2}
\end{equation}
for some constant $C$, where $\nu = ak/h$ is the Courant number.

\item Conclude from the above steps that $B$ is uniformly power bounded and
hence the leapfrog method is Lax-Richtmyer stable provided that $|\nu|<1$.

\end{enumerate} 

\item Show that the leapfrog method with periodic boundary conditions is
also stable in the case $|ak/h|=1$ if $m+1$ is not divisible by 4. 
Find a good set of initial data $U^0$ and $U^1$ to illustrate the
instability that arises if $m+1$ is divisible by 4 and perform a calculation
that demonstrates nonconvergence in this case.

\end{enumerate}

