\documentstyle[11pt]{article}
\parindent 0.5in
\setlength{\oddsidemargin}{-0.35in}
\setlength{\textwidth}{7in}
\topmargin -.7in
\textheight 10.0in
\renewcommand{\baselinestretch}{1}

\begin{document}

\begin{center}
\section*{Free advice on time management\\ 
\large for new faculty and graduate students\\[0.02in]
\large Greg Conradi Smith}
written a decade ago, compiled on \today
\end{center}

\begin{itemize}
\item As much as possible, work through students, and learn how to manage students.  Construct your lab's culture so that student success and your success are achieved in parallel.  For example, you might insist that graduate students write papers and publish them before writing their dissertation.  You might insist that learned students help train new students, providing opportunities for your students to teach. 

\item Screen students interested in undergraduate research experiences carefully.  Make sure that REU students demonstrate interest somehow before you commit.  For example, you might insist that prospective REU students complete a long, relevant self-paced tutorial; or attend group meetings; or review the literature; etc.   Don't give time-sensitive projects to undergraduates. 

\item Avoid faculty politics, especially if you find it emotionally challenging.  

\item Communicate via email, phone, or  visiting as appropriate. 

\item Find a pace of work that suits you.  Protect the time when you are most creative and reserve that time for creative work.   For example, if reading is relaxing, you can read in the evenings.  If you write well in the morning, work at home or in a cafe in the morning. 

\item Be careful to distinguish ``urgent'' versus ``important'' matters.  Do not let the ``urgent'' displace the truly ``important,'' but do not let simple ``urgent'' tasks accumulate.  Do ``important'' things when you are ``strong.''   Do ``urgent'' things when you are ``weak.''   Primary relationships are ``important.'' 

\item Remember that to a large extent faculty write their own job description.   As far as it depends on you, teach courses you like to teach, learn what types of service you enjoy, etc.  

\item Understand your department's expectations with regard to research and teaching.  You are your own person, but it is wise to understand exactly how you will be evaluated.  

\item As quickly as possible, learn to whom you are allowed to say ``no.'' 

\item Learn to recognize when you are ``spinning your wheels'' and take a break or change tasks when you notice this.

\item When writing long documents do the following in order:
\begin{itemize}
\item Review literature and think about what you want to write
\item Outline your document
\item Stock the sections
\item Quickly work your way through entire document from start to finish
\item Make final decision about organization of the paper
\item Do final editing 
\end{itemize}
It is inefficient to edit your writing before the overall structure and content of the document is clear. 

\item  Avoid perfectionism.  Except when it comes to editing grant proposal and papers. 



\end{itemize}

\end{document}

