
\exercise[(codes for heat equation)]{9.2}

\begin{enumerate} 
\item The m-file \verb+heat_CN.m+ solves
the heat equation $u_t = \kappa u_{xx}$ using the Crank-Nicolson method.
Run this code, and by changing the number of grid points, confirm that it is
second-order accurate.  (Observe how the error at some fixed time such as $T=1$
behaves as $k$ and $h$ go to zero with a fixed relation between $k$ and $h$,
such as $k = 4h$.)

You might want to use the function \verb+error_table.m+ to print out this
table and estimate the order of accuracy, and \verb+error_loglog.m+ to
produce a log-log plot of the error vs.\ $h$.  See \verb+bvp_2.m+ for an
example of how these are used.

\item Modify \verb+heat_CN.m+ to produce a new m-file \verb+heat_trbdf2.m+ that
implements the TR-BDF2 method on the same problem.  Test it to confirm that
it is also second order accurate.  Explain how you determined the proper
boundary conditions in each stage of this Runge-Kutta method.

\item Modify \verb+heat_CN.m+ to produce a new m-file \verb+heat_FE.m+ that
implements the forward Euler explicit 
method on the same problem.  Test it to confirm that
it is $\bigo(h^2)$ accurate as $h\goto 0$ provided when $k = 24 h^2$ is
used, which is within the stability limit for $\kappa = 0.02$.  Note how
many more time steps are required than with Crank-Nicolson or TR-BDF2,
especially on finer grids.

\item Test \verb+heat_FE.m+ with $k = 26 h^2$, for which it should be
unstable.  Note that the instability does not become apparent until about
time 1.6 for the parameter values $\kappa = 0.02,~ m=39,~\beta = 150$.
Explain why the instability takes several hundred time steps to appear, and
why it appears as a sawtooth oscillation. 

{\bf Hint:} What wave numbers $\xi$ are growing exponentially for these
parameter values?  What is the initial magnitude of the most unstable
eigenmode in the given initial data?  The expression (16.52) for the Fourier
transform of a Gaussian may be useful.

\end{enumerate}

